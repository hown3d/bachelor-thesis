\section{Externe Apps}
\label{sec:komponenten:externe-apps}

Um den Chat Service effektiv und einfach im Cluster zu betreiben, bringt das Kubernetes Ökosystem viele hilfreiche Anwendungen mit.
Hierzu werden die folgenden Kubernetes nativen Anwendungen genutzt:
\begin{itemize}
  \item ExternalDNS
  \item Nginx Ingress Controller
  \item Cert-Manager
\end{itemize}

Zusätzlich kommen mit dem Nutzen des \acf{SCS} (siehe \ref{sec:gaia-x-einbettung:scs}),
welcher auf OpenStack basiert, der Service \textbf{Designate}
für DNS Konfiguration sowie \textbf{Cinder} als Storage Service dazu. 
Designate und die Kubernetes Anwendung \textbf{ExternalDNS} werden miteinander
verknüpft, um für neue Ingress Objekte des Service automatisch DNS Einträge zu erstellen. 

Der \textbf{Nginx Ingress Controller} dient als Reverse Proxy für alle Ingress Ressourcen innerhalb des Clusters.
Mithilfe des in \ref{sec:komponenten:operator} genannten Synchronisationsmechanismus wird der Nginx Webserver so 
konfiguriert, sodass mittels Host Headern in Anfragen an den Server die jeweils richtige Route zu den Containern hergestellt wird \cite{nginxIngressController}. 
Da der Controller als Proxy dient, muss zudem nur eine externe IP und damit ein Loadbalancer für alle Chat Services genutzt werden.
Dies führt zu Kostenersparnissen in der Cloud für den Provider und dadurch auch für den Kunden.
Die DNS Einträge für alle Chat Instanzen werden durch die Nutzung 
von ExternalDNS ebenfalls automatisch an die externe IP des Nginx Controller Loadbalancers angepasst \cite{externalDNS}.

Um verschlüsselte TLS Verbindungen zu nutzen, wird die von JetStack entwickelte Anwendung \textbf{Cert-Manager} genutzt,
welche die Erstellung von SSL Zertifikaten für Ingress Ressourcen vereinfacht. Im Cluster wird ein \emph{Issuer} Objekt
erstellt, welche als Zertifizierungsstelle für neue Zertfikate dient \cite{CertManager2021}. 
Ein Issuer unterstützt dabei mehrere Arten zur Erstellung neuer Zeritifkate.
wobei für die Referenzimplementation die Entscheidung zunächst auf ein selbstsigniertes Zertifikat, dass neue Zertfikate signiert, gefallen ist \cite{CertManager2021}.
Eine Umstellung auf einen andere Art, wie beispielweise ACME\footnote{Automated Certificate Managment Environment},
ist jederzeit möglich. 
Beim Erstellen einer Ingress Ressource wird die Annotation \texttt{cert-manager.io/issuer} mit dem gewünschten Issuer angehängt,
auf welche die Cert-Manager Anwendung ein Zertifikat am Issuer erstellt \cite{CertManager2021}. 
Das in der Ingress Ressource referenzierte TLS Secret wird daraufhin vom Cert-Manager für das Zertifikat genutzt \cite{CertManager2021}.

\subsection{Identität- und Zugriffsmanagement}
Kubernetes besitzt mit dem \ac{RBAC} System eine Möglichkeit innerhalb des Clusters Zugriffsrechte auf Objekte
zu verwalten. In dieser Referenzimplementation soll nun ein außenstehender Dienst für das Identitätsmanagment
zuständig sein. Hierbei wurde auf Basis der \ac{SCS} Architektur \emph{Keycloak} als OpenID Connect Provider gewählt.
OpenID Connect ist eine Schicht überhalb des OAuth Frameworks, welches Clients die Möglichkeit bietet, Endnutzer
an einer Authentifizierungsstelle zu verifizieren und Informationen über diese zu erhalten \cite{OpenID2021}.


Kubernetes unterstützt ebenfalls das OpenID Connect Protokoll, indem Optionen für die OpenID Connect Authentifizierungsstelle
übergeben werden. Dadurch ist es für den Endnutzer möglich, sich mittels eines \emph{OpenID Connect Tokens} am 
Kubernetes API Server zu authentifizieren. Die Nutzung dieses Prinzip für den Chat Service wird 
in \ref{sec:komponenten:rocket-api-service} und \ref{sec:komponenten:tenant-service} näher erläutert.
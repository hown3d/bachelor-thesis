\section{Tenant Service}
\label{sec:komponenten:tenant-service}
Zur Konfiguration des Tenants im Clustern, wurde der Tenant Service als Referenzimplementation erstellt. Hierbei
wird ebenfalls das gRPC Framework verwendet, um eine konistente API zu ermöglichen. Im Fall des Tenant Service
besitzt die API nur einen \emph{Register} Endpoint. Dieser stellt sicher, dass für den gewünschten Nutzer 
Kubernetes Roles und Rolebindings, ein Cert-Manager Issuer, sowie Resource Quotas angelegt wird. 
Da Rollen an einen Namespace gebunden sind, erhält jeder Tenant einen eigenen Namespace, um sicherzustellen,
dass andere Nutzer keinen Zugriff auf die Ressourcen einer anderen Partei erlangen.
Diese werden vom Rocket API Service genutzt, um die entsprechenden Rocket \acp{CR} zu managen. 
Im Gegensatz zum Rocket API Service, verwendet der Tenant Service nicht das Prinzip der Identitätsübernahme.
Der Tenant Service läuft neben Operator und Rocket API Service im Kubernetes Cluster, stellt allerdings seine
Anfragen an die Kubernetes API mit Hilfe eines Service Accounts. Dieser Service Account besitzt die selben Rechte,
wie die erstellten User Rollen, da Kubernetes nicht erlaubt, Rechte zu vergeben, wenn die anfragende Identität
diese nicht besitzt.
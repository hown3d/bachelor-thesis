\section{Tenant Service}
\label{sec:komponenten:tenant-service}
Zur Konfiguration des Tenants im Cluster, wurde der Tenant Service zur Verwaltung von Nutzerkonfiguration im Cluster erstellt.
Hierbei wird das gRPC Framework verwendet, um eine konistente API zu ermöglichen. Im Fall des Tenant Service
besitzt die API nur einen \emph{Register} Endpoint. Dieser stellt sicher, dass für den gewünschten Nutzer 
Kubernetes Roles und Rolebindings, ein Cert-Manager Issuer, sowie Resource Quotas angelegt wird. 
Da Rollen an einen Namespace gebunden sind, erhält jeder Tenant einen eigenen Namespace, um sicherzustellen,
dass andere Nutzer keinen Zugriff auf die Ressourcen einer anderen Partei erlangen.
Jene Namespaces werden vom Rocket API Service genutzt, um die entsprechenden Rocket \acp{CR} der Nutzer zu verwalten. 

Im Gegensatz zum Rocket API Service, verwendet der Tenant Service nicht das Prinzip der Identitätsübernahme, da
der Service administrative Bereitstellungen für den Nutzer im Cluster vornimmt. 
Diese Bereitstellung findet mit Hilfe eines Service Accounts, welcher als Identität für Anfragen 
an die Kubernetes API dient, statt . Dieser Service Account besitzt die selben Rechte,
wie die erstellten User Rollen, da Kubernetes nicht erlaubt, Rechte zu vergeben, wenn die anfragende Identität
diese nicht besitzt.


\chapter{Service Level Agreements}
\label{chapter:sla}
Endnutzer von Software sind auf dem Weg zum vermehrten Nutzung von \ac{SaaS} Produkten, wodurch
die Qualität und Verfügbarkeit dieser Services ein wichtiger Aspekt sind. Da allerdings die 
Anforderungen der Nutzer sich unterscheiden und es keinen \emph{One Size fits all} Ansatz gibt,
müssen Anbieter und Nutzer gemeinsame Einigungen bezüglich der Erwartungen an den Service getroffen werden.
Um diese Einigungen einzuhalten, muss der Anbieter die Qualitätsanforderungen der Anwendung überwachen \cite{Patel2009}.
In dieser Referenzimplementation soll auf die Basis Qualitätsanforderungen des Chat Services eingegangen werden,
sowie dessen Monitoring mittels freier Software aus der \ac{SCS} Architektur. 
\section{Kubernetes}
\label{sec:grundlagen:kubernetes}
Seit der Einführung von Google in 2014 hat sich Kubernetes zu einer der größten und populärsten 
Open Source Projekte der Welt entwickelt \cite{Burns2019}. Es ist zum Standard für die Entwicklung von cloud-Native 
Anwendungen geworden \cite{Burns2019}. 

Kubernetes ist ein System, mit dem eine große Anzahl an Containeranwendungen einfach bereitgestellt und verwaltet werden können \cite{Marko2018}.
Im System werden alle Anwendungen als deklarative Konfigurationsobjekte dargestellt, welche einen gewünschten
Status des Systems abbilden. Kubernetes muss dabei sicherstellen, dass der tatsächliche Status dem gewünschten entspricht \cite{Burns2019}.
Dabei ist es ein \emph{Selbstheilendes System}.
Das bedeutet, das System überwacht den aktuellen Status kontinuierlich, um Abweichungen zu erkennen \cite{Burns2019}.
Als anwendungsorientierte Container-API kann Kubernetes eine Abstraktion von spezifischen \acp{VM} für Entwickler kreieren \cite{Burns2019}.
Entwickler müssen nicht wissen, auf welcher \ac{VM} ihre Anwendung bereitgestellt wird.
Maschinen können somit einfach dem Cluster bei Notwendigkeit hinzugefügt werden \cite{Burns2019}.
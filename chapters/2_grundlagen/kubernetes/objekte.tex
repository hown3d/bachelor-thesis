\subsection{Objekte}
\label{sec:kubernetes:objekte}
In Kubernetes sind Objekte persistente Entitäten des Systems. 
Das System nutzt diese Entitäten um den Status des Clusters zu repräsentieren.

Objekte können die folgenden Dinge beschreiben:
\begin{itemize}
  \item Welche containerisierten Anwendungen laufen auf welchen Knoten im Cluster?
  \item Welche Ressourcen benötigt die Anwendung im Cluster?
  \item Richtlinien, wie sich die Anwendungen verhalten soll, beispielsweise Ausfallsicherheit, Updates und Neustarts
\end{itemize}

Sobald ein Objekt erstellt wird, arbeitet Kubernetes daran, das die gewünschte Konfiguration des Objekts
im Cluster vorhanden ist. 
Effektiv bedeutet die Erstellung eines Objekts die Definition des Zustands im Clusters \cite{KuberneteObjects_2021}.

Um Objekte innerhalb eines Clusters in Gruppen zu isolieren, bietet Kubernetes das Prinzip des \textbf{Namespace} \cite{kubernetesNamespaces}.
Durch die Nutzung von Namespaces können komplexe Systeme mit unzähligen Komponenten
in kleine, deutliche Gruppen geteilt werden \cite{Marko2018}.
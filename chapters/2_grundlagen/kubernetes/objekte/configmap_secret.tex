\subsubsection{ConfigMaps und Secrets}
\label{subsec:kubernetes:configmap_secret}
ConfigMaps enthalten Konfigurationsinformationen für Anwendungen. 
Diese Informationen können fein granuliert vorliegen, beispielsweise als eine Anzahl mehrerer Strings
oder als zusammengesetzter Wert in Form einer Datei \cite{Burns2019}.
Secrets sind ähnlich zu ConfigMaps, werden jedoch eher für die Bereitstellung von sensitiven Informationen wie
SSL Zertfikaten oder Zugangsdaten verwendet \cite{Burns2019}.
Da Secrets unabhängig von der Anwendung, dem Pod, erstellt werden,
besteht ein geringeres Risiko, dass das Secret (und seine Daten) während des Arbeitsablaufs 
beim Erstellen, Anzeigen und Bearbeiten von Pods preisgegeben werden \cite{kubernetesSecrets}. 
Für Secrets als auch ConfigMaps gibt es drei Hauptarten, um diese als Konfiguration für Pods zu nutzen:
\begin{description}
  \item [Dateisystem] Informationen, wie beispielsweise Dateien werden in das Dateisystem des Containers eingehängt \cite{Burns2019}.
  \item [Umgebungsvariablen] Erzeugen dynamisch Umgebungsvariablen innerhalb des Containers basierend auf dem Inhalt des Secrets/ConfigMap \cite{Burns2019}.
  \item [Kommandozeilenargumente] Kubernetes bietet die Möglichkeit, Kommandozeilenargumente mithilfe einer Secret/ConfigMap zu erstellen \cite{Burns2019}.
\end{description}
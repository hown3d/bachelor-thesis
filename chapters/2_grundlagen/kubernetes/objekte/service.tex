\subsubsection{Service}
\label{subsec:kubernetes:service}
Pods können innerhalb eines Clusters auf mehreren Knoten verteilt operieren.
Beispielsweise erstellt und löscht Kubernetes bei der Nutzung eines Deployments dynamisch Pods \cite{kubernetesService}.
Da jeder Pod eine eigene IP-Adresse innerhalb des Clusters zugeteilt bekommt, führt das zum Problem der Kommunikation innerhalb des Systems.
Welche IP-Adressen der Pods (beispielsweise Backend),
die eine Funktionalität für andere Pods bereitstellen, soll von anderen Pods (beispielsweise Frontend) verwendet werden \cite{kubernetesService}?
\paragraph{Service Discovery}
dient als Lösung für dieses Problem.
Grundlage der Service Discovery sind in Kubernetes \textbf{Service} Objekte.
In Kubernetes sind Services eine Abstraktion, welche eine Gruppe von Pods und eine Richtlinie für den Zugriff auf diese definiert \cite{kubernetesService}.
Jeder Service besitzt einen sogenannten \emph{Selector}, über welchen definiert wird, welche Pods in die Gruppe des Services aufgenommen werden sollen \cite{Burns2019}.
Zur Kommunikation von anderen Pods werden Service Objekten eine eigene virtuelle IP-Adresse 
sowie ein DNS Eintrag innerhalb des Clusters zugewiesen.
Über diese IP-Adresse werden Anfragen an Pods über einen Lastverteiler, welcher durch die \emph{kube-proxy} Komponente gesteuert wird,
verteilt \cite{Burns2019}.
Um Clients nicht mit virtuellen, sich möglicherweise ändernden IP-Adressen zu konfigurieren, stellt Kubernetes einen DNS Service zur
Verfügung \cite{Burns2019}.
Der DNS-Name eines Services setzt sich immer aus dem Namen und dem Namespace des Services zusammen \cite{kubernetesService}.
\subsubsection{Volume}
\label{subsec:kubernetes:volume}
Das Dateisystem in einem Container ist flüchtig, 
sodass neue Dateien innerhalb eines Containers nach Neustart nicht mehr vorhanden sind \cite{kubernetesVolumes}.
In manchen Fällen ist es gewünscht, produzierte Daten zu speichern.
Kubernetes ermöglicht dies durch den Einsatz von \textbf{Volumes} \cite{Marko2018}.
Im Grunde sind Volumes ein Verzeichnis auf das Container innerhalb des Pods zugreifen können \cite{kubernetesVolumes}.
Volumes sind an den Lebenszyklus eines Pods gebunden. Falls ein Pod gelöscht wird, beispielsweise durch eine neue Version des Deployment,
wird das Volume ebenfalls gelöscht.
Wenn eine containerisierte Anwendung peristenten Speicher auf einer Festplatte benötigt und dieser Speicher auch bei neuen Pods verfügbar sein soll,
werden \acfp{PV} genutzt \cite{Marko2018}.

\paragraph{Persistent Volumes}
bieten eine Abstraktion für Benutzer und Administratoren der Speicherbereitstellung und -nutzung sowie
die unterliegende Infrastrukturspezifikation des Speichers.
\acp{PV} können ebenfalls als Volumes genutzt werden,
sind jedoch vom Lebenszyklus des Pods unabhängig \cite{kubernetesPV}.

\paragraph{Persistent Volume Claims}
sind Anfragen von Nutzern auf Speicher. 
Sobald ein neuer \ac{PVC} im Cluster erstellt wird, provisioniert oder findet Kubernetes das
passende \ac{PV} und bindet es an den Claim.
Innerhalb eines Pod Templates können \acp{PVC} als Volumes genutzt werden.
Sobald ein Pod ein \ac{PV} als Volume nutzt wird es für andere Pods gesperrt. 
Dies wird realisiert durch einen Verbund von \ac{PV} und \ac{PVC} \cite{Marko2018}.

\subsubsection{Rollenbasierte Zugriffskontrolle}
\label{subsec:kubernetes:rbac}
Rollenbasierte Zugriffskontrolle bietet einen Mechanismus zur Zugriffs- und Aktionsbeschränkung auf Kubernetes API Endpunkte \cite{kubernetesRbac}.
Durch Nutzung dieser Kontrolle wird sichergestellt, dass nur vorgesehene Nutzer Zugriff auf gewisse API Endpunkte und damit deren
Objekte besitzen \cite{Burns2019}.

Unterschieden werden im \ac{RBAC} System zwischen vier Kubernetes Objekten:
\begin{itemize}
  \item Role
  \item ClusterRole
  \item RoleBinding
  \item ClusterRoleBinding
\end{itemize}

Objekte mit dem Prefix \emph{Cluster} sind nicht auf einen Namespace im Cluster beschränkt \cite{kubernetesRbac}.
\emph{Role} und \emph{ClusterRole} Objekte enthalten Regeln, die eine Gruppe von Berechtigungen enthält \cite{kubernetesRbac}.

\begin{lstlisting}[caption=Repräsentation eines Role Objekts in YAML, label=listing:role_yaml]
apiVersion: rbac.authorization.k8s.io/v1
kind: Role
metadata:
  namespace: default # Begrenzt auf den Namespace "default"
  name: pod-reader
rules:
- apiGroups: [""] # "" indicates the core API group
  resources: ["pods"]
  verbs: ["get", "watch", "list"]
\end{lstlisting}

Als Beispiel wird in \ref{listing:role_yaml} die Repräsentation eines Role Objekts dargestellt. 
Jeder Nutzer, der dieser Rolle besitzt, kann Pod Objekte im Namespace \emph{default} lesen.


Um nun eine solche Role einem Nutzer oder \emph{Service Account} zuzuweisen,
werden \emph{RoleBinding} und \emph{ClusterRoleBinding} Objekte genutzt \cite{Burns2019}.

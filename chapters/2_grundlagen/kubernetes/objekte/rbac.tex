\subsubsection{Rollenbasierte Zugriffskontrolle}
\label{subsec:kubernetes:rbac}
Rollenbasierte Zugriffskontrolle bietet einen Mechanismus zur Zugriffs- und Aktionsbeschränkung auf Kubernetes API Endpunkte.
Durch Nutzung dieser Kontrolle wird sichergestellt, dass nur vorgesehene Nutzer Zugriff auf gewisse API Endpunkte und damit deren
Objekte besitzen \cite{Burns2019}.

Unterschieden werden im \ac{RBAC} System zwischen vier Kubernetes Objekten:
\begin{itemize}
  \item Role
  \item ClusterRole
  \item RoleBinding
  \item ClusterRoleBinding
\end{itemize}

Objekte mit dem Prefix \emph{Cluster} sind nicht auf einen Namespace im Cluster beschränkt \cite{kubernetesRbac}.
\emph{Role} und \emph{ClusterRole} Objekte enthalten Regeln, die eine Gruppe von Berechtigungen enthält \cite{kubernetesRbac}.
Um eine solche Role nun einem Nutzer oder \emph{Service Account} (\ref{subsec:kubernetes:serviceaccount}) zuzuweisen,
werden \emph{RoleBinding} und \emph{ClusterRoleBinding} Objekte genutzt \cite{Burns2019}.

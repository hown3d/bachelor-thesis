\section{Software as a Service}
\label{sec:grundlagen:saas}
Durch die Nutzung eines \acf{SaaS} werden Kunden eine Standardsoftwarelösung als Dienstleistung via Internet bereitgestellt.
Der Anbieter des \ac{SaaS} ist dabei für den Betrieb und Wartung der Anwendung verantwortlich \cite{Buxmann2008}.
Hauptgrund der Nutzung eines \ac{SaaS} ist die Einsparung von Kosten.
Die Einsparungen werden durch den Wegfall von Investitionen in Lizenzen und Infrastrukturen realisiert,
da die Gebühren eines \ac{SaaS} meistens als \emph{Pay per Use} Modell umgesetzt werden \cite{Tan2013}.
Weitere Vorteile der Nutzung eines \ac{SaaS} sind \textbf{Zugänglichkeit, Zuverlässigkeit, Konfigurierbarkeit} und \textbf{Skalierbarkeit} \cite{Tan2013}.
\paragraph{Zugänglichkeit}
wird durch einen einfachen Zugang zum Service gewährleistet. Nutzer können über Browser auf die Anwendung zugreifen \cite{Tan2013}.

\paragraph{Zuverlässigkeit}
Da die Wartung des \ac{SaaS} durch den Anbieter erfolgt, ist er ebenfalls verantwortlich für Verfügbarkeit und Sicherheit des Service.
Um Datenverlust zu vermeiden, können Backups der Nutzerdaten erstellt werden \cite{Tan2013}.

\paragraph{Konfigurierbarkeit}
Jeder Nutzer des Service kann den Dienst an ihre individuellen Bedürfnisse anpassen.
Realisiert wird dies zwecks der \emph{Multitenancy} Fähigkeit eines \ac{SaaS} \cite{Tan2013}.
Tenants sind Interessengruppen oder Firmen, die eine maßgeschneiderte Anwendung des \ac{SaaS} Anbieters mieten \cite{Schroeter2012}.
Multitenancy ermöglicht es einem \ac{SaaS} mehrere Kundenkonfigurationen innerhalb des selben Service einzusetzen \cite{Tan2013}.

\paragraph{Skalierbarkeit}
Cloud Computing bietet die Möglichkeit, Rechenkapazität auf das Bedürfnis der Nutzer anzupassen \cite{Tan2013},
ohne die Architektur der Anwendung zu ändern \cite{Satyanarayana2012}.
Aufgrund dessen muss die Architektur des \ac{SaaS} in der Lage sein, Lastverteilung über mehrere identische Anwendungsinstanzen
zu unterstützen \cite{Satyanarayana2012}.
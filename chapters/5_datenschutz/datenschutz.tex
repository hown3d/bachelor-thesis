\chapter{Datenschutz}
\label{chapter:datenschutz}
Hyperscaler Clouds wie \ac{AWS}, Microsoft Azure oder \ac{GCP} unterliegen dem amerikanischen \textbf{US Cloud Act}, welcher
den Datenschutz in Europa gespeicherter Daten gefährdet \cite{Kagermann2021}. Der US Cloud Act besagt,
dass amerikanische Cloud Provider verpflichtet sind, jeglichen Datenverkehr zu einem Kunden zu speichern und 
US-Behörden Zugriff auf jene Daten zu gewährleisten \cite{CloudAct2018}.
Aufgrund dieses Gesetzes besteht beim Cloud Computing in amerikanischen Clouds ein grundlegendes Problem 
zur \emph{Integrität und Vertraulichkeit der Datenverarbeitung} \cite{Weichert2010}.


Gaia-X stellt in \cite{Gaia-XPolicy} Richtlinien auf, welche jedes Service Angebot, das mit dem Gaia-X Framework bereitgestellt wird,
befolgen muss. 
Dabei sind im Punkt Datenschutz die Richtlinien des Artikel 26 und 28 der Datenschutz-Grundverordnung einzuhalten \cite{Gaia-XPolicy}.
Im Kern müssen die folgenden Punkte der beiden DS-GVO Artikel abgedeckt sein \cite{Gaia-XPolicy}:
\begin{itemize}
  \item Service Provider sind streng an die Kundenvorgaben gebunden
  \item Klare Definition wie der Kunde Vorageben erstellt (mittels API oder Konfigurationstool)
  \item Bei Weitergabe von Daten an Drittanbieter muss dies explizit angeben, sowie getreu Artikel 5 der DS-GVO geschützt werden
  \item Klare Definition inwiefern Unterauftragsverarbeiter der Daten involviert sind
\end{itemize}


\section{Persistenz von Daten im Kubernetes Cluster}
\label{subsec:datenschutz:persistenz}
Innerhalb des Kubernetes Clusters werden alle Chatnachrichten innerhalb der vom Operator bereitgestellten MongoDB gespeichert.
Als Persistenzebene der MongoDB wird ein \ac{PV} genutzt,
welches als Abstraktion für Speicherimplementierungen für Kubernetes dient.
Als Implementierung dieser Abstraktion wird 
Cinder\footnote{Block Storage Service von OpenStack} \ac{CSI}\footnote{\href{https://github.com/kubernetes/cloud-provider-openstack/tree/master/pkg/csi/cinder}{Cinder CSI}} genutzt. 
Eine Implementation des \ac{CSI} muss hierbei die folgenden Aktionen unterstützen \cite{container-storage-interface_2021}:
\begin{itemize}
  \item Dynamische Provisionierung sowie Löschen von Volumes
  \item Anhängen oder Abtrennen von Volumes an einem Knoten
  \item Einhängen und Aushängen von Volumes an einem Knoten.
  \item Verwaltung von sowohl Block- als auch einhängbarer Volumes
  \item Erstellung und Löschen von Snapshots (Quelle eines Snapshots ist immer ein Volume)
  \item Erstellen eines neuen Volumes von einem Snapshot
\end{itemize}

Sobald im Cluster ein \ac{PVC} erstellt wird, kümmert sich das \ac{CSI} mittels dem Cinder Service von OpenStack 
um die Bereitstellung von Speicher innerhalb der Plusserver Cloud \cite{cinderCSI}.
Hierbei werden die Volumes immer in der selben Region wie die Knoten des Kubernetes Cluster erstellt \cite{kubernetesVolumes}.
Der Nutzer des Services kann also sicher sein, dass die Daten des genutztes Chat Services immer in der zuvor 
gewählten Region verwahrt werden.
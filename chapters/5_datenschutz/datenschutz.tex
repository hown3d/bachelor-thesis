\chapter{Datenschutz}
\label{chapter:datenschutz}
Hyperscaler Clouds wie \ac{AWS}, Microsoft Azure oder \ac{GCP} unterliegen dem amerikanischen \textbf{US Cloud Act}, welcher
den Datenschutz in Europa gespeicherter Daten gefährdet \cite{Kagermann2021}. Der US Cloud Act besagt hierbei,
dass amerikanische Cloud Provider verpflichtet sind, jeglichen Datenverkehr zu einem Kunden zu speichern und 
US-Behörden Zugriff auf jene Daten zu gewährleisten \cite{CloudAct2018}.
Aufgrund dieses Gesetzes besteht beim Cloud Computing ein grundlegendes Problem 
zur \emph{Integrität und Vertraulichkeit der Datenverarbeitung} \cite{Weichert2010}.


\section{Persistenz von Daten}
\label{subsec:datenschutz:persistenz}
Innerhalb des Kubernetes Clusters werden alle Chatnachrichten innerhalb der vom Operator bereitgestellten
MongoDB gespeichert. Im Kubernetes Cluster sind alle Pods der MongoDB Container mithilfe eines StatefulSets 
erstellt. Zur Persistenz wird das vom StatefulSet verwendet PersistVolume genutzt,
welches als Abstraktion für Speicherimplementierungen für Kubernetes dient.
Als Implementierung dieser Abstraktion wird 
Cinder\footnote{Block Storage Service von OpenStack} CSI\footnote{\href{https://github.com/kubernetes/cloud-provider-openstack/tree/master/pkg/csi/cinder}{Cinder CSI}} genutzt. 
Eine Implementation des \ac{CSI} muss hierbei die folgenden Aktionen unterstützen \cite{container-storage-interface_2021}:
\begin{itemize}
  \item Dynamische Provisionierung sowie Löschen von Volumes
  \item Anhängen oder abtrennen von Volumes an einem Knoten
  \item Einhängen und Aushängen von Volumes an einem Knoten.
  \item Verwaltung von sowohl Block- als auch einhängbarer Volumes
  \item Erstellung und Löschen von Snapshots (Quelle eines Snapshots ist immer ein Volume)
  \item Erstellen eines neuen Volumes von einem Snapshot
\end{itemize}

Sobald im Cluster ein \ac{PVC} erstellt wird, kümmert sich die Anwendung um das Bereitstellen von 
Cinder Volumes innerhalb der Cloud mittels dem Cinder Service von OpenStack \cite{cinderCSI}.
Hierbei werden die Volumes immer in der selben Cloudumgebung wie die Knoten erstellt, wodurch der 
Nutzer des Services sicherstellen kann, dass die Daten des genutztes Chat Services immer in der zuvor 
gewählten Region verwahrt werden.
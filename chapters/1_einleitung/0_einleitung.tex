\chapter{Einleitung}
\label{chap:einleitung}

% Context, Motiviation, Lösungsweg, erzielte Ergebnisse, weiterer Aufbau

% Context
Das Projekt \emph{Gaia-X} wurde als Basis für eine föderale, offene Dateninfrastruktur auf der Grundlage europäischer Werte gegründet.
Das Ziel der Initiative ist die Erstellung einer föderalisierten Dateninfrastruktur, 
welche auf europäischen Werten zu Daten und Cloudsouveränität basiert \cite{GaiaXArchitecture2021}.
Die europäische Cloud soll als vernetzte, europäische Alternative zu etablierten \emph{Hyperscaler Clouds}, wie \ac{AWS},
Microsoft Azure und \ac{GCP} dienen. Ziel ist eine homogene und standardisierte Dateninfrastruktur für europäische Cloudanbieter.
Das Konzept der Gaia-X Cloud ist ein vernetztes Knotenprinzip, in dem Transparenz mittels einer Selbstbeschreibung der Services
über Ort der Datenspeicherung und der verwendenten Technologie geboten wird.
Die Teilnahme von Cloudanbietern an dem Projekt erfordert die Einhaltung gewisser Standards, 
sowie die Erfüllung von Gaia-X Richtlinien durch eine Zertifizerung \cite{BMWi2019}.


\ac{SaaS} Anwendungen in Gaia-X sollen mithilfe eines Katalogs für Nutzer verfügbar gemacht werden \cite{GXFS2021}.
\ac{SaaS} ist ein Cloud-Dienst Angebot, welches die Bereitstellung und Verwaltung von betriebsbereiten, 
gehosteten Anwendungen verspricht \cite{Krebs2012}.
Da die funktionalen und nicht-funktionalen Anforderungen von \emph{Tenants} sich unterscheiden können,
müssen Anwendungen konfigurierbar sein \cite{Schroeter2012}. 
Moderne container-basierte Anwendungsplattformen vereinfachen das Management und die Entwicklung
von Software. Kubernetes \cite{kubernetes} ist eine Open Source Containerorchestrierungsplattform und gilt im
Cloud Native Umfeld als defacto Standard API für den Betrieb von Cloud-native Anwendungen \cite{Burns2019}.
Es bietet die Möglichkeit, skalierbare und verteilte Systeme zu erstellen und zu warten \cite{Burns2019}.
Kubernetes Operatoren können die Kubernetes API durch eigene Ressourcen zu erweitern. 
Dieses Prinzip wurde 2016 von CoreOS\footnote{\url{https://coreos.com}} erstmals vorgestellt.

% Motiviation
\paragraph{}
Chat-Systeme, wie Rocket.Chat oder Slack, werden bereits in vielen Unternehmen ausschließlich oder im Zusammenhang mit E-Mails
zur Kommunikation innerhalb des Unternehmens genutzt. Daten dieser Chat-Systeme sind aus diesem Grund oftmals vertraulich
und können Firmengeheimnisse sowie personenbezogene Daten enthalten. 
Amerikanische Cloudanbieter unterliegen dem amerikanischen \textbf{US Cloud Act}, welcher
den Datenschutz in Europa gespeicherter Daten gefährdet \cite{Kagermann2021}.
Betreibt ein Unternehmen nun ein Chat-System bei einem amerikanischen Cloud-Provider oder nutzt eine existierende
\ac{SaaS} Lösung wie \emph{Slack.com}, kann dies zu einer Gefährdung des Datenschutzes führen.
Als europäische Alternative zu etablierten Anbietern soll Gaia-X Abhilfe für dieses Problem schaffen.
Dabei befindet sich die europäische Cloud aktuell noch im Entwicklungsstadium und ist dadurch nicht vollständig einsatzbereit.
Aus diesem Grund soll in dieser Arbeit ein Ansatz für containerbasierte \ac{SaaS} Anwendungen in der Gaia-X Cloud gegeben werden.
% \citeauthor{Truyen2016} geben in \cite{Truyen2016} eine mögliche Architektur für die Entwicklung eines \acf{SaaS} vor.
% Die Architektur berücksichtigt allerdings nicht den Einsatz des Operator Patterns, 
% dass die Bereitstellung und Verwaltung der Softwarekomponenten deutlich vereinfacht \cite{Dobies2020}.

% Lösung
\paragraph{}
Diese Arbeit dient als Referenzimplementation für einen 
Rocket.Chat \ac{SaaS}, welche auf dem Operator Pattern von Kubernetes basiert sowie mehrere \emph{Middleware Komponenten} nutzt.
Rocket.Chat ist dabei ein frei verfügbares Chatsystem, dass DSGVO-Konformität und Sicherheit verspricht \cite{RocketChat}.
Der \ac{SaaS} wird in Kubernetes betrieben und für Endnutzer als zentrale Schnittstelle angeboten.
Dazu wird eine lokale Testinfrastruktur mittels dem Framework OpenStack erstellt,
welches als Basis für Gaia-X Services dient \cite{scs}.

% Ergebnisse
\paragraph{}
Ein produktiver Einsatz neuer Services in Gaia-X kompatiblen Clouds hat sich als noch nicht einsatzbereit erwiesen.
Zu viele Komponenten des Gaia-X Standards sind bisher nur als Definition im Architekturbild vorhanden, sodass eine konkrete,
nutzbare Implementation dieser Standards aus bleibt. 
Testimplementationen, wie beispielsweise die Plusserver Open Cloud, geben aktuell nur die Möglichkeit,
Infrastrukturkomponenten zu nutzen.
Services für Kunden anzubieten, ist durch den fehlenden Standard von Gaia-X zur Zeit nur über den traditionellen Weg eines
\ac{SaaS} möglich.
Ein Softwarekatalog für Gaia-X Services ist zwar geplant \cite{BMWi2019},
wird aktuell allerdings noch nicht angeboten.

% weiterer Aufbau
\paragraph{}
Im weiteren Verlauf dieser Arbeit werden einige Grundlagen bezüglich Container und Kubernetes gegeben,
sowie eine Vertiefung zum Thema Gaia-X und \acf{SaaS}.
Es folgt die Erläuterung der einzelnen Komponenten der Referenzimplementation, sowie die Datensicherheit in Gaia-X.
Kapitel \ref{chapter:gaia-x-einbettung} beschreibt, wie der Einsatz mit einer Gaia-X kompatiblen Umgebung ermöglicht wird.
Zuletzt wird in Abschnitt \ref{chapter:fazit} ein Fazit, Ausblick für die Zukunft, sowie ein Vergleich zu verwandten Arbeiten gegeben.
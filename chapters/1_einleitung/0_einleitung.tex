\chapter{Einleitung}
\label{chap:einleitung}

Moderne container-basierte Anwendungsplattformen vereinfachen das Managment und die Entwicklung
von Software. Kubernetes \cite{kubernetes} ist eine Open Source Containerorchestrierungsplattform und gilt im
Cloud Native Umfeld als defacto Standard API für den Betrieb von Cloud-native Anwendungen.
Es bietet die Möglichkeit, skalierbare und verteilte Systeme zu erstellen und zu warten \cite{Burns2019}.
Mithilfe von Kubernetes Operatoren, einem Pattern, welches von CoreOS\footnote{\url{https://coreos.com}} entwickelt,
ist es möglich, die Kubernetes API mit eigenen Ressourcen zu erweitern.
Das Projekt \emph{Gaia-X} wurde als Basis für eine föderale, offene Dateninfrastruktur auf der Grundlage europäischer Werte gegründet.
Als Hauptziel gilt die digitale Souveränität, innerhalb der Plattform sollen Daten sicher gesammelt und verteilt werden können.
Die europäische Cloud soll als vernetzte europäische Alternative zu etablierten \emph{Hyperscaler Clouds}, wie Amazons \ac{AWS},
Microsoft Azure und Googles \ac{GCP}. Ziel ist eine homogenen und standartisierten Dateninfrastruktur für europäische Cloudanbieter.
Das Konzept für die Gaia-X Cloud ist ein vernetztes Knotenprinzip, indem über eine Selbstbeschreibung der Knoten
Tranzparenz über Ort der Datenspeicherung und der verwendenten Technologie bietet. Um an dem Projekt als Cloudanbieter
teilnehmen zu können, muss ein gewisser Standard sowie die Erfüllung von Regeln durch eine Zertifizierung nachgewiesen werden.\cite{BMWi2019}

\paragraph{}
Diese Arbeit soll eine Referenzimplementation für einen \emph{Multi-Tenant}\footnote{Teilen eines Systems mit anderen Nutzern, welche spezifische Privilegien innerhalb des Systems besitzen}
Chat \ac{SaaS} bieten, welche auf dem Operator Pattern von Kubernetes aufbaut.
% und wie \acp{SLA} \ref{chapter:sla} eingehalten werden.
Eine Einbettung in bereits vorhandene Gaia-X Testumgebungen soll ebenfalls realisiert und verdeutlicht werden,
wie bereits jetzt \acp{SaaS} in Gaia-X umsetzbar sind. Dazu wird eine lokale Teststruktur mittels dem Framework OpenStack erstellt,
welches für Gaia-X auch genutzt werden soll \cite{scs}.

Zunächst werden einige Grundlagen bezüglich Container und Kubernetes gegeben, sowie eine Vertiefung zum Thema Gaia-X.
Im Verlauf der Arbeit werden desweiteren die einzelnen Komponenten der Referenzimplementation erläutert, sowie die Einhaltung der \acp{SLA}.
Im Kapitel \ref{chapter:gaia-x-einbettung} wird beschrieben, wie der Einsatz mit einer Gaia-X kompatiblen Umgebung ermöglicht wurde.
Zuletzt werden in Abschnitt \ref{chapter:fazit} schließlich einige abschließende Bemerkungen gemacht.
\section{Motivation}
\label{sec:einleitung:motivation}
Hyperscaler Clouds wie \ac{AWS}, Microsoft Azure oder \ac{GCP} unterliegen dem amerikanischen \textbf{US Cloud Act}, welcher
den Datenschutz in Europa gespeicherter Daten gefährdet \cite{Kagermann2021}, weshalb im Jahr 2019 das Projekt Gaia-X angekündigt wurde.
Gaia-X befindet sich aktuell noch im Entwicklungsstadium und ist dadurch nicht vollständig einsatzbereit.
Es existieren Testimplementationen, wie beispielsweise die Plusserver Open Cloud, welche jedoch nur Infrastruktur-
komponenten, den \ac{IaaS} Teil der Cloud, bereitstellen und keine Services anbieten. Ein konsistenter Softwarekatalog für Gaia-X ist zwar geplant\cite{BMWi2019},
wird aktuell allerdings noch nicht angeboten. Aus diesem Grund soll für zukünfigte Projekte in der Gaia-X Cloud ein Ansatz
für containerbasierte \ac{SaaS} Anwendungen erstellt werden. 
\ac{SaaS} Applikationen sind dabei eine weitverbreitete Methode im Cloud-Computing Umfeld \cite{Schroeter2012}.
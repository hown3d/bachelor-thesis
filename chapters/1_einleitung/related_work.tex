\section{Related Work}
In \cite{Truyen2016} beschreiben \citeauthor{Truyen2016}
das Erstellen von \emph{Multi-Tenant}\footnote{Teilen eines Systems mit anderen Nutzern, welche spezifische Privilegien innerhalb des Systems besitzen}
\acf{SaaS} mithilfe von Container Orchestrierungssoftware wie Kubernetes.
\acp{SaaS} werden immer häufiger auf skalierbaren Plattformen gehostet, um eine hohe Verfügbarkeit sowie Ausfallsicherheit zu gewährleisten.
Innerhalb ihres Papers gehen sie auf die Vor- und Nachteile des Containeransatzes ein. 
Ein klarer Vorteil sei die Möglichkeit des Multi-Cloud Modells, da Kubernetes unabhängig der jeweiligen Cloud Anbieter einsetzbar ist.
und einen Standard implementieren.
Isolationsmöglichkeiten zwischen den \emph{Tenants} habe einen hohen Anspruch für Sicherheit und Datenschutz.
Ein weiterer Problemfall sei der hohe Aufwand zum Managment des Deployments der Anwendung.
Sie präsentieren eine Referenzarchitektur zum Erstellen von Container basierten Multi-Tenant SaaSs.
Einen fehlenden Standard für Container Images bemängeln sie zum Zeitpunkt ihrer Veröffentlichung.
Als weiteren Verbesserungspunkt stellen \citeauthor{Truyen2016} 3 Möglichkeiten zur Isolation von Tenants vor.

\paragraph{}
Als weitere Möglichkeit zur Abstraktion von Softwaremanagment für Endnutzer entwickeln \citeauthor{Wieder2012} in ihrem Paper
\cite{Wieder2012} ein System, das die Cloud-Kunden von der Last befreit, zu entscheiden, welche Dienste innerhalb einer Cloud zu verwenden.
Dabei wird speziell auf den Fall von MapReduce Operationen eingegangen. Kunden sollen nur entsprechende Ziele definieren, beispielsweise
die Minimierung der Kosten, eine Berechnung, die in der Cloud ausgeführt werden soll, sowie eine Liste von Cloud Services.
Das System erstellt automatisch die Berechnung und optimiert während der Laufzeit die Anwendung basierend auf externen Zuständen.

\paragraph{}
Im Paper \cite{Bousselmi2014} wird das \ac{CPP} für SaaSs beschrieben, welches als Hauptziel die Optimierung der Ressourcennutzung
der Cloudkomponenten sowie die Minimerung der Antwortzeit verfolgt.
Eine gelungene Lösung für das \ac{CPP} solle folgende Anforderungen erfüllen:
\begin{itemize}
  \item \texttt{Platzierungsalgorithmus}
  \item \texttt{Infrastrukturbeschreibung}
  \item \texttt{Provisionierungsalgorithmus}
  \item \texttt{Cloudunabhängig}
\end{itemize}
Desweiteren bieten \citeauthor{Bousselmi2015} eine Übersicht der bereits existierenden Lösungen des \ac{CPP} für \acp{SaaS}.
\section{Erstellung der Testinfrastruktur}
\label{sec:gaia-x-einbettung:erstellung-testinfra}
Bei der Erstellung der Testinfrastruktur kam es zu einigen Einschränkungen im Vergleich zu bereits etablierten Hyperscaler Clouds,
da zum Zeitpunkt der Thesis nur Infrastruktur Services wie Netzwerke, DNS, \acp{VM} und Speicher
und kein Managed Kubernetes Service zur Verfügung standen.
Durch die unterliegende Architektur des Gaia-X Providers kann allerdings bereits etablierte Software
zur Erstellung von Kubernetes Clustern genutzt werden.
Dazu wurde das Tool \textbf{Kubespray}\footnote{\url{https://github.com/kubernetes-sigs/kubespray}} genutzt,
welches mit Hilfe der Konfigurationssoftware \textbf{Ansible} und der \ac{IaC} Software \textbf{Terraform}
\acp{VM}, Netzwerk und Sicherheitsgruppen für die Kubernetes Knoten in der Cloud erstellt.

Aufgrund der zeitlichen Begrenzung der Testlizenz auf 30 Tage wurde jedoch zusätzlich eine lokal
testbare Version der benötigten OpenStack Services erstellt.
Alle benötigen Services, sowie eine vollwertiges Kubernetes Cluster, werden in Containern bereitgestellt,
sodass sie lokal betrieben werden können.
Somit kann die Cloudunabhängigkeit von Kubernetes zum Vorteil für Entwickler und der schnellen Entwicklung genutzt werden.
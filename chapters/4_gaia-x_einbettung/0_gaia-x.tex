\chapter{Einbettung in Gaia-X kompatible Cloud}
\label{chapter:gaia-x-einbettung}
Als einer der ersten Testimplementationen für Gaia-X gilt die Pluscloud Open.
Dabei wurde zum Zweck dieser Arbeit eine zeitlich begrenzte Testlizenz bereitgestellt, 
welche die Möglichkeit bat, diese Referenzimplementation eines \ac{SaaS} in Gaia-X zu testen.


\section{Erstellung der Testinfrastruktur}
\label{sec:erstellung-testinfra}
Hierbei kam es allerdings zu einigen Einschränkungen im Vergleich zu bereits etablierten Clouds, 
wie \ac{AWS}, Azure \dots, da zum Zeitpunkt der Thesis nur Infrastruktur Services wie Netzwerke, DNS, \acp{VM} und Speicher 
zur Verfügung stehen und kein Managed Kubernetes Service. Um dennoch ein Kubernetes Cluster zu erstellen,
wurde das Tool \textbf{Kubespray}\footnote{\href{https://github.com/kubernetes-sigs/kubespray}{Kubespray}} genutzt,
welches mit Hilfe Red Hats Konfiugrationssoftware \textbf{Ansible} und Hashicorps \textbf{Terraform} \acp{VM}, Netzwerk und Sicherheitsgruppen
für die Kubernetes Knoten erstellt. Dadurch war es relativ einfach, ein Cluster für Testzwecke zu erstellen.

Aufgrund der zeitlichen Begrenzung von 30 Tagen der Testlizenz, wurde jedoch noch eine lokal 
testbare Version der Openstack Services erstellt, welche Plusserver nutzt.
Hierbei werden alle benötigen Services in Containern bereitgestellt, welche auf der Entwicklungs-
plattform betrieben werden können. 
Somit kann die Cloudunabhängigkeit von Kubernetes zum Vorteil für Entwickler genutzt werden.


\section{Zukünfigtige Bereitstellung im Gaia-X Katalog}
\label{sec:erstellung-testinfra}
Gaia-X definiert als Standard einen sogenannten Katalog, in dem Services registriert werden können und Endnutzer
mittels eines Suchalogrithmuses einen Service für ihre Bedürfnisse finden können. 
Im Katalog bereitgestellte Services müssen den Nutzer informieren, welche Eigenschaften der jeweilige Service besitzt.
Dadurch können Endnutzer auswählen, welche Services in welcher Region und zu welchen Umständen sie nutzen möchten.
Dieses Prinzip wird in Gaia-X \emph{Self-Description} genannt. Self-Descriptions werden mittels JSON-LD dargestellt, ein
leichtgewichtiges \emph{Linked Data} Format. \cite{Eggers2020}

Zum Zeitpunkt der Thesis ist dieser Katalog jedoch rein theoretisch, sodass noch keine Beispiele für diesen Katalog existieren.

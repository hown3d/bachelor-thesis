\section{Sovereign Cloud Stack}
\label{sec:gaia-x-einbettung:scs}

\begin{figure}[h]
  \centering
  \includegraphics[height=0.71\textwidth]{gfx/chapters/4_gaia-X/scs_architecture.png}
  \caption{Architektur und Komponenten des Sovereign Cloud Stacks}
  \source{\cite{scs}}
  \label{fig:scs_architecture}
\end{figure}

\ac{SCS} ist ein Open-Source Projekt, um eine standartisierte und souveräne Plattform zu definieren, welche von 
existierenden und zukünfitgen Cloudprovidern genutzt werden soll. 
Ziel des Stacks ist ein Netzwerk von Anbietern, welche durch Nutzung von freier Software und gemeinsamer Standards,
eine interoperable, unabhängige Cloud schaffen \cite{Kagermann2021}.
Als technologischer Standpunkt dient \ref{fig:scs_architecture}, welches die geplante Architektur des \ac{SCS} zeigt. 
Grundlage des Stacks sind OpenStack Services, welche als OpenSource Projekt für Cloud-Computing Architekturen entwickelt wurden.
Unterteilt wird dies in drei grundlegende Bausteine: \textbf{Compute}, \textbf{Network} und \textbf{Storage},
welche als \ac{IaaS} Services definiert sind.
Die Openstack Services sollen als starke, multitenant fähige Basis für Kubernetes Cluster dienen. 
Der Hauptdienst soll Kubernetes as a Service darstellen, auf dem Provider intern ihre \ac{SaaS} aufbauen können \cite{scs}.
Darauf aufbauend wird ein Container Layer erstellt, welcher mit Hilfe von Containerruntimes
wie Docker oder 
Podman\footnote{Daemonlose Containerlaufzeitumgebung zur Verwaltung, Erstellung und Betrieb von Containern \cite{podman}.}
gesteuert werden soll \cite{scs}.

Services wie der in dieser Thesis entwickelte Chat \ac{SaaS} finden sich auf der übergeordneten Ebene \textbf{SCS Platform Services} wieder.
Für die Entwicklung des RocketChat \ac{SaaS} wurde diese Architektur als Grundbild genutzt, indem 
genannte Technologien des Stacks in der Implementierung berücksichtigt wurden.
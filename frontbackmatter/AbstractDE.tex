%*******************************************************
% Abstract in German
%*******************************************************
\begin{otherlanguage}{ngerman}
	\pdfbookmark[0]{Zusammenfassung}{Zusammenfassung}
	\chapter*{Zusammenfassung}

	Große public Cloud Anbieter wie AWS, Google und Microsoft haben es bei Datenschützern schwer:
	Wie werden sensible Daten, wie es Chat Nachrichten sein können, DSGVO konform gespeichert.
	Für dieses Problem wurde die Initiative Gaia-X gegründet, um einen europäischen Cloud Service anzubieten.

	Gaia-X besitzt seit 2021 mit Plusserver Open Cloud, die erste Referenzcloud für die zukünfigte Gaia-X Architektur,
	wobei aktuell nur Infrastrukturlösungen, wie \acp{VM}, Speicher und Netzwerk angeboten werden. 
	Software Services, wie amerikanische Hyperscaler Clouds anbieten, sind aktuell noch nicht implementiert.
	Deshalb soll in dieser Arbeit eine neue Referenzimplementation für einen Chat \ac{SaaS} in einer Gaia-X kompatiblen Cloud erstellt werden.
	Bei einem Software as a Service Prinzip soll der Endnutzer sich nicht um den Betrieb, Bereitstellung der Software
	oder Skalierung der Anwendung kümmern müssen. 
	Die \ac{SaaS} Anwendung wird als containerbasierte Lösung erstelt, welche mittels dem defacto Standard Framework für 
	Containerorchestrierung Kubernetes \cite{Burns2019} betrieben wird. Vorteil dieses Ansatzes ist die einfache Skalierungsmöglichkeit,
	ein großes Ökosystem für Kubernetes sowie Unabhängigkeit von Cloudanbietern, 
	sodass diese Architektur auf jeder Infrastruktur zum Einsatz kommen kann \cite{Burns2019}.

	Das Ergebnis dieser Arbeit zeigt, dass ein \ac{SaaS} für Gaia-X schon heute möglich ist, 
	wobei der produktive Einsatz und damit als Konkurrenz für etablierte amerikanische Cloudanbieter noch nicht erreichbar ist. 
	Gaia-X definiert zwar einen Standard zur Erstellung eines förderalisierten Systems und zur Kommunikation und Interaktion 
	der Teilnehmer, jedoch ist dieser noch nicht effektiv einsetzbar. 

\end{otherlanguage}
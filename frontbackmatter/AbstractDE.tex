%*******************************************************
% Abstract in German
%*******************************************************
\begin{otherlanguage}{ngerman}
	\pdfbookmark[0]{Zusammenfassung}{Zusammenfassung}
	\chapter*{Zusammenfassung}

	Die Marktführer im Public Cloud Bereich (Amazon, Google und Microsoft) aus dem US-amerikanischen Raum
	haben es bei Datenschützern schwer:
	Wie werden sensible Daten, wie es Chat Nachrichten sein können, DS-GVO konform gespeichert.
	Dies ist eine Frage des europäischen Datenschutzes und dem Konflikt zu US-amerikanischen Gesetzen wie dem US Cloud Act.
	Für dieses Problem wurde die Initiative Gaia-X gegründet, um einen europäischen Cloud Service anzubieten.

	In vielen Unternehmen werden Chat-Systeme bereits als zentrales Kommunikationssystem eingesetzt.
	Daten in solchen Chat-Systemen sind daher oftmals vertraulich und können Firmengeheimnisse enthalten.
	Betreibt ein Unternehmen ein solches System bei einem amerikanischen Cloudanbieter, stellt sich die Frage nach Datensicherheit.
	Deshalb soll in dieser Arbeit eine neue Referenzimplementation für einen Chat \acf{SaaS} in einer Gaia-X kompatiblen Cloud erstellt werden,
	welcher die genannten Datenschutzprobleme beseitigt.
	Bei einem \ac{SaaS} Prinzip soll der Endnutzer sich nicht um den Betrieb, Bereitstellung der Software
	oder Skalierung der Anwendung kümmern müssen. 
	Die \ac{SaaS} Anwendung wird als containerbasierte Lösung erstellt, welche mittels dem defacto Standard Framework für 
	Containerorchestrierung Kubernetes \cite{Burns2019} betrieben wird. Vorteil dieses Ansatzes ist die einfache Skalierungsmöglichkeit,
	ein großes Ökosystem für Kubernetes sowie Unabhängigkeit von Cloudanbietern, 
	sodass diese Architektur auf jeder Infrastruktur zum Einsatz kommen kann \cite{Burns2019}.
	\paragraph{}
	Gaia-X besitzt seit 2021 mit Plusserver Open Cloud, die erste Referenzcloud für die zukünfigte Gaia-X Architektur,
	in der diese \ac{SaaS} Anwendung zum Einsatz kommen kann.	
	Plusserver Open Cloud bietet jedoch aktuell nur Infrastrukturlösungen, wie \acp{VM}, Speicher und Netzwerk an. 
	Software Services, wie amerikanische Cloudprovider anbieten, sind aktuell noch nicht implementiert.
	Das Ergebnis dieser Arbeit zeigt, dass ein \ac{SaaS} für Gaia-X schon heute möglich ist, 
	wobei der produktive Einsatz noch nicht empfehlenswert ist.
	Eine Konkurrenz zu etablierten amerikanischen Cloudanbietern ist Gaia-X noch nicht. 
	Gaia-X definiert zwar einen Standard zur Erstellung eines föderalisierten Systems und zur Kommunikation und Interaktion 
	der Teilnehmer, jedoch ist dieser noch nicht vollständig implementiert. 

\end{otherlanguage}
%*******************************************************
% Abstract in German
%*******************************************************
\begin{otherlanguage}{ngerman}
	\pdfbookmark[0]{Zusammenfassung}{Zusammenfassung}
	\chapter*{Zusammenfassung}

	Große public Cloud Anbieter wie AWS, Google und Microsoft haben es bei Datenschützern schwer:
	Wie werden sensible Daten, wie es Chat Nachrichten sein können, DSGVO konform gespeichert.
	Für dieses Problem wurde die Initiative Gaia-X gegründet, um einen europäischen Cloud Service anzubieten.


	Gaia-X besitzt seit 2021 mit Plusserver Open Cloud, die erste Referenzcloud für die zukünfigte Gaia-X Architektur,
	wobei aktuell nur Infrastrukturlösungen, wie \acp{VM}, Speicher und Netzwerk angeboten wird. 
	Software Services, wie amerikanische Hyperscaler Clouds anbieten, sind aktuell noch nicht implementiert.
	Deshalb soll in dieser Arbeit eine neue Referenzimplemntation für einen \ac{SaaS} Chat in einer Gaia-X kompatiblen Cloud erstellt werden.
	Bei einem Software as a Service Prinzip soll der Endnutzer sich nicht um den Betrieb, Bereitstellung der Software
	oder Skalierung der Anwendung kümmern müssen. Für den Endnutzer wird eine Schnittstelle bereitgestellt, 
	worüber er seinen Service erstellen kann.
	Als Implementation des \ac{SaaS} wird eine containerbasierte Lösung angestrebt, welche mittels dem defacto Standard Framework für 
	Containerorchestrierung Kubernetes \cite{Burns2019} betrieben wird. Vorteil dieses Ansatzes ist die einfache Skalierungsmöglichkeit,
	ein großes Ökosystem für Kubernetes sowie Unabhängigkeit von Cloudanbietern, 
	sodass diese Architektur auf jeder Infrastruktur zum Einsatz kommen kann \cite{Burns2019}.

	Ein \ac{SaaS} für Gaia-X ist also schon heute möglich und kann damit zur Konkurrenz für etablierte amerikanische Cloudanbieter werden,
	wobei dieses Ziel noch weit in der Zukunft liegt. Aktuell bietet Gaia-X eine \ac{IaaS} Struktur an,
	welche sich für den produktiven Einsatz im Gegensatz zu AWS, Azure und Google Cloud nicht anbietet,
	wenn mehr als nur \acp{VM} benötigt werden.
\end{otherlanguage}

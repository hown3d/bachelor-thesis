%*******************************************************
% Abstract in English
%*******************************************************
\pdfbookmark[0]{Abstract}{Abstract}


\begin{otherlanguage}{american}
	\chapter*{Abstract}
The market leaders in the public cloud sector (Amazon, Google and Microsoft) from the U.S. region
	have a hard time with data protectionists:
	How is sensitive data, such as chat messages, stored in compliance with the GDPR?
	This is a question of European data protection and the conflict with US law such as the US Cloud Act.
	For this problem, the Gaia-X initiative was founded to provide a European cloud service.

	In many companies, chat systems are already used as a central communication system.
	Data in such chat systems is therefore often confidential and may contain company secrets.
	If a company operates such a system with an American cloud provider, the question of data security arises.
	Therefore, a new reference implementation for a chat \ac{SaaS} in a Gaia-X compatible cloud is to be created in this work.
	In a Software as a Service principle, the end-user should not have to worry about the operation, deployment
	or scaling of the application. 
	The \ac{SaaS} application is created as a container based solution using the defacto standard framework for 
	Kubernetes container orchestration \cite{Burns2019}. The advantage of this approach is the ease of scaling,
	a large ecosystem for Kubernetes as well as independence from cloud providers, 
	so that this architecture can be deployed on any infrastructure \cite{Burns2019}.
	\paragraph{}
	Gaia-X has Plusserver Open Cloud, the first reference cloud for the future Gaia-X architecture, since 2021,
	in which this SaaS application will be deployed.	
	However, Plusserver Open Cloud currently only offers infrastructure solutions such as \acp{VM}, storage and network. 
	Software services, like American hyperscaler clouds offer, are not currently implemented.
	The result of this work shows that a \ac{SaaS} for Gaia-X is already possible today, 
	although productive use and thus as competition for established American cloud providers is not yet achievable. 
	Gaia-X defines a standard for the creation of a federalized system and for the communication and interaction of the participants. 
	of the participants, but this cannot yet be used effectively. 


\end{otherlanguage}
